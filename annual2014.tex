%%--------------- Text starts from here ----------- %

%%%%%%%%%%%%%%%2014年度Annual Report用 スタイルファイル%%%%%%%%%%%%%%%%%%%%%%%%
%このFormatはpLaTex を使用しています.
%以下に報告書の基本形が示してありますので,参考にしてお書き下さい. 
%数字は2桁以上は全て半角で書いて下さい. 
%文末の空白は必ず半角でお願いします.全角の空白は TeX では特殊文字と 
%判断して問題を起こすことがあります.
%数式はかならずmath mode でお願いいたします. 
%事務局では校正をせずにprint out したものをそのまま印刷に回しますので, 
%一度コンパイルして,スペルチェック,校正は必ず行なって下さい. 
%まとめの編集の都合上,\newcommand, \renewcommand, \def の追加等はさけて下さいま%すようお願いいたします.



\documentclass[a4j,twocolumn]{jarticle}

\usepackage{amssymb,amsmath}
\textheight=25cm
\textwidth=15cm
\parskip=0mm
\parindent=0mm
\topmargin=-1cm
\oddsidemargin=5mm

\begin{document}

%%%%%%%%%%%%%%%%%%%%%%%%%%%%%%%%%%%%%%%%%%%%%%%%%%%%%%%%%%%%%%%%%%%%%%%%%%% 
% 新しい連絡先
% 2015年度から所属や連絡先が変わる予定の人はここにご記入下さい
%%%%%%%%%%%%%%%%%%%%%%%%%%%%%%%%%%%%%%%%%%%%%%%%%%%%%%%%%%%%%%%%%%%%%%%%%%% 


%%%%%%%%%%%%%%%%%%%%%%%%%%%%%%%%%%%%%%%%%%%%%%%%%%%%%%%%%%%%%%%%%%%%%%%%%%% 
% 2014年度 東大数理における身分
% 該当するものだけを残し,それ以外の行を削除して下さい
%%%%%%%%%%%%%%%%%%%%%%%%%%%%%%%%%%%%%%%%%%%%%%%%%%%%%%%%%%%%%%%%%%%%%%%%%%% 
修士課程学生 (Master's Course Student)
%%%%%%%%%%%%%%%%%%%%%%%%%%%%%%%%%%%%%%%%%%%%%%%%%%%%%%%%%%%%%%%%%%%%%%%%%%% 


%%%%%%%%%%%%%%%%%%%%%%%%%%%%%%%%%%%%%%%%%%%%%%%%%%%%%%%%%%%%%%%%%%%%%%%%%%%% 
% 氏名(ローマ字綴りで名字は全て大文字,名前は最初の字だけ大文字) 
% を書いて下さい.
%{\bf 数理 太郎 (SURI Taro)}
% というふうに.
%%%%%%%%%%%%%%%%%%%%%%%%%%%%%%%%%%%%%%%%%%%%%%%%%%%%%%%%%%%%%%%%%%%%%%%%%%% 

{\bf 高橋 和音 (TAKAHASHI Kazune)}

%%%%%%%%%%%%%%%%%%%%%%%%%%%%%%%%%%%%%%%%%%%%%%%%%%%%%%%%%%%%%%%%%%%%%%%%%%%
% 索引用データ
% 一人あたりAlphabet順索引・五十音順索引用に
% 2行必要です.
%
% 日本人 
% \index{アルファベット表記 (日本語表記)}
% \index{かな読み@日本語表記}
% 例
% \index{TSUBOI Takashi (坪井 俊)}
% \index{つぼいたかし@坪井 俊}
%
% 外国人(漢字表記なし)
% \index{アルファベット表記}
% \index{かな読み@カタカナ表記}
% 例
% \index{SUTHICHITRANONT Noppakhun}
% \index{すってぃちとらのん@スッティチトラノン ノッパクン}
%
% 外国人(漢字表記あり)
% \index{アルファベット表記 (漢字表記)}
% \index{かな読み@カタカナ表記}
% 例
% \index{LI Xiaolong (李 曉龍)}
% \index{りしゃおろん@リ シャオロン}
%
%%%%%%%%%%%%%%%%%%%%%%%%%%%%%%%%%%%%%%%%%%%%%%%%%%%%%%%%%%%%%%%%%%%%%%%%

\index{TAKAHASHI Kazune (高橋 和音)}
\index{たかはしかずね@高橋 和音}

\vspace{0.2cm}
\noindent
A. 研究概要

\vspace{0.1cm}
%%%%%%%%%%%%%%%%%%%%%%%%%%%%%%%%%%%%%%%%%%%%%%%%%%%%%%%%%%%%%%%%%%%%%%%%%% 
% 研究の要約を日本語で,その下に英訳をつけて書いて下さい.
%%%%%%%%%%%%%%%%%%%%%%%%%%%%%%%%%%%%%%%%%%%%%%%%%%%%%%%%%%%%%%%%%%%%%%%%%% 

%和文%

楕円型偏微分方程式の解の存在・非存在を,
変分法を用いて研究している.
本年度は,ソボレフ臨界指数を持つ非斉次半線形楕円型偏微分方程式
$-\Delta u + a u = b u^p + \lambda f$を考察した.
特に,次元と解の存在・非存在の関係について研究した.
$b$が領域の内点で最大値を達成し,その点の
近傍で$a$が指数$q$の増大度を持つ
とき,領域の次元が$6 + 2q$未満ならば,少なくとも$2$つの正値解が
存在することを証明した.線形項の係数が解が存在する領域の次元に
影響するのは,新しい現象であると思われる.

\vspace{0.5cm}
%英文%

I study the existence and nonexistence of the solutions
of elliptic PDEs using the variational method.
In this academic year, I worked on the following
nonhomogeneous semilinear elliptic equation
involving the critical Sobolev exponent:
$-\Delta u + a u = b u^p + \lambda f$. Especially, I studied
the relationship between the dimension of the domain and
the existence and nonexistence of the solutions.
I proved that provided 
$b$ acheves its maximum at an inner point of the
domain and $a$ has a growth of the exponent $q$
in some neighborhood of that point, then
if the dimension of the domain is less than $6 + 2q$,
there exist at least two positive solutions.
It seems to be new that the coefficient of a linear term affects
the dimension of the domain under which solutions exist.

%\\%
\vspace{0.2cm}


\noindent
B. 発表論文

\vspace{0.1cm}
%%%%%%%%%%%%%%%%%%%%%%%%%%%%%%%%%%%%%%%%%%%%%%%%%%%%%%%%%%%%%%%%%%%%%%%%%%%%%% 
% 5年以内(2010\UTF{FF5E}2014年度)10篇まで書いて下さい.但し,2014年1月1日\UTF{FF5E} 
% 2014年12月31日に出版されたものは,10篇を超えてもすべて含めて下さい.
% 様式は以下の例のように
% 著者・共著者名・ \lq\lq 題名・ジャーナル名・巻・年・ページの順に書いて下さい.
% タイトルの前に著者・共著者名を入れる形です.
% 共著の場合 T. Katsura and #.####などと書きwith 共著者名とはしない様に
% お願い致します.
%%%%%%%%%%%%%%%%%%%%%%%%%%%%%%%%%%%%%%%%%%%%%%%%%%%%%%%%%%%%%%%%%%%%%%%%%%%%% 

\begin{enumerate}
\item K. Takahashi: \lq\lq Semilinear elliptic equations with critical
      Sobolev exponent and non-homogeneous term",
      東京大学修士論文 (2015).
\end{enumerate}

\end{document} 